% Copyright 2012 Wenjun Deng <wdeng@wdeng.info>
%
% This file is part of PIC1D-PETSc
%
% PIC1D-PETSc is free software: you can redistribute it and/or modify
% it under the terms of the GNU General Public License as published by
% the Free Software Foundation, either version 3 of the License, or
% (at your option) any later version.
%
% PIC1D-PETSc is distributed in the hope that it will be useful,
% but WITHOUT ANY WARRANTY; without even the implied warranty of
% MERCHANTABILITY or FITNESS FOR A PARTICULAR PURPOSE.  See the
% GNU General Public License for more details.
%
% You should have received a copy of the GNU General Public License
% along with PIC1D-PETSc.  If not, see <http://www.gnu.org/licenses/>.


% PIC1D-PETSc formulation document
\documentclass[12pt]{article}

\usepackage[colorlinks,linkcolor=blue,citecolor=blue,bookmarks,pdfstartview=FitH]{hyperref}
% amsmath is necessary for correct parenthesis around equation number
\usepackage{amsfonts,amsmath}

% \Autoref is for the beginning of the sentence
\let\orgautoref\autoref
\providecommand{\Autoref}[1]
{%
    \def\equationautorefname{Equation}%
    \def\figureautorefname{Figure}%
    \def\subfigureautorefname{Figure}%
    \def\sectionautorefname{Section}%
    \orgautoref{#1}%
}
% \autoref is used inside the sentence to produce Fig., and Eq. for figures,
% subfigures, and equations
\renewcommand{\autoref}[1]
{%
    \def\equationautorefname{Eq.}%
    \def\figureautorefname{Fig.}%
    \def\subfigureautorefname{Fig.}%
    \def\sectionautorefname{Sec.}%
    \def\subsubsectionautorefname{Sec.}%
    \orgautoref{#1}%
}

% Put parenthesis around equation numbers
\makeatletter
\def\tagform@#1{\maketag@@@{\ignorespaces#1\unskip\@@italiccorr}}
\let\orgtheequation\theequation
\def\theequation{(\orgtheequation)}
\makeatother

\begin{document}
\newcommand{\md}{\mathrm{d}}
\newcommand{\me}{\mathrm{e}}

\title{1D electrostatic $\delta f$ particle-in-cell (PIC) formulation in vector-matrix form \\
\large for \texttt{PIC1D-PETSc version 2012-08-24 22:36:24-04:00}}
\author{Wenjun Deng}
\maketitle


%%%%%%%%%%%%%%%%%%%%%%%%%%%%%%%%%%%%%%%%
\section{1D Vlasov-Poisson equations}

\begin{eqnarray}
	\frac{\md f_\alpha(x, v, t)}{\md t} & = & \partial_t f_\alpha(x, v, t) + v \partial_x f_\alpha(x, v, t) + \frac{Z_\alpha E(x, t)}{m_\alpha} \partial_v f_\alpha(x, v, t) \nonumber \\
	& = & 0 \textrm{ ,} \\
	\partial_x E(x, t) & = & 4 \pi \sum_\alpha Z_\alpha n_\alpha(x, t) \textrm{ ,} \\
	n_\alpha(x, t) & \equiv & \int_{-\infty}^{\infty} f_\alpha(x, v, t) \md v \textrm{ .}
\end{eqnarray}
$\alpha$ indicates species.
$f$ is particle distribution.
$Z$ is charge.
$m$ is mass.
$n$ is density.
$E$ is electrostatic field.


%%%%%%%%%%%%%%%%%%%%%%%%%%%%%%%%%%%%%%%%
\section{$\delta f$ formulation}

Decompose quantities into equilibrium part and perturbed part:
\begin{eqnarray}
	f_\alpha & = & f_{0 \alpha} + \delta f_\alpha \textrm{ ,} \\
	n_\alpha & = & n_{0 \alpha} + \delta n_\alpha \textrm{ ,} \\
	n_{0 \alpha} & \equiv & \int_{-\infty}^{\infty} f_{0 \alpha} \md v \textrm{ ,} \\
	\delta n_{\alpha} & \equiv & \int_{-\infty}^{\infty} \delta f_{\alpha} \md v \textrm{ .}
\end{eqnarray}
Note that equilibrium electrostatic field is zero, $E$ is perturbed field.

Assume equilibrium distribution is Maxwellian, which is time-independent and uniform in $x$:
\begin{eqnarray}
	f_{0 \alpha} & = & f_{0 \alpha}(v) = \frac{n_{0 \alpha}}{\sqrt{2 \pi v_{\mathrm{th}, \alpha}^2}} \exp \left( \frac{v^2}{2 v_{\mathrm{th}, \alpha}^2} \right) \textrm{ ,} \\
	v_{\mathrm{th}, \alpha} & \equiv & \frac{T_\alpha}{m_\alpha} \textrm{ ,} \\
	\partial_t f_{0 \alpha} & = & 0 \textrm{ ,} \\
	\partial_x f_{0 \alpha} & = & 0 \textrm{ .}
\end{eqnarray}

$\delta f$ formulation:
\begin{eqnarray}
	\frac{\md}{\md t} \delta f & = & - \frac{\md}{\md t} f_0 \nonumber \\
	& = & - \frac{Z_\alpha}{m_\alpha} E \partial_v f_{0 \alpha} \textrm{ ,} \\
	\partial_x E & = & 4 \pi \sum_\alpha Z_\alpha \,\delta n_\alpha \textrm{ .}
\end{eqnarray}


%%%%%%%%%%%%%%%%%%%%%%%%%%%%%%%%%%%%%%%%
\section{$\delta f$ particle-in-cell (PIC) formulation}
\label{sec:df_pic}

Physical particle distribution is carried by marker particles.
Each marker particle carries 4 quantities in this case: $x$, $v$, $p$ and $w$.
Marker particle equilibrium weight $p$ and perturbed weight $w$ are defined as:
\begin{eqnarray}
	p & \equiv & \frac{f_{0 \alpha}}{g_\alpha} \textrm{ ,} \\
	w & \equiv & \frac{\delta f_\alpha}{g_\alpha} \textrm{ ,} \\
\end{eqnarray}
where $g$ is marker particle distribution function:
\begin{equation}
	g_\alpha(x, v, t) = \sum_{j_\alpha = 1}^{N_\alpha} S_x(x - x_{j_\alpha}) S_v(v - v_{j_\alpha}) \textrm{ .}
\end{equation}
$N_\alpha$ is number of marker particles for species $\alpha$.
$j_\alpha$ is marker particle index.
$x_{j_\alpha}$ and $v_{j_\alpha}$ are position and velocity of the $j_{\alpha}$-th marker particle, respectively.
$S_x$ and $S_v$ are shape functions in $x$ and $v$, respectively.
In this PIC formulation, take $S_v$ to be Dirac $\delta$ function and $S_x$ to be hat function:
\begin{eqnarray}
	S_v(v - v_{j_\alpha}) & = & \delta(v - v_{j_\alpha}) \textrm{ ,} \\
	S_x(x - x_{j_\alpha}) & = & \left\{ \begin{array}{ll}
		\frac{h_x - |x - x_{j_\alpha}|}{h_x^2} & |x - x_{j_\alpha}| < h_x \\
		0 & \textrm{otherwise}
	\end{array} \right. \textrm{ ,}
\end{eqnarray}
where $h_x$ is grid size in $x$.

Physical particle distribution functions are given by:
\begin{eqnarray}
	f_\alpha(x, v, t) & = & \sum_{j_\alpha = 1}^{N_\alpha} p_{j_\alpha} S_x(x - x_{j_\alpha}) S_v(v - v_{j_\alpha}) \textrm{ ,} \\
	\delta f_\alpha(x, v, t) & = & \sum_{j_\alpha = 1}^{N_\alpha} w_{j_\alpha} S_x(x - x_{j_\alpha}) S_v(v - v_{j_\alpha}) \textrm{ .}
\end{eqnarray}

Note that for Maxwellian:
\begin{equation}
	\frac{\partial_v f_{0 \alpha}}{f_{0 \alpha}} = - \frac{v}{v_{\mathrm{th}, \alpha}^2} = - \frac{v}{T_\alpha / m_\alpha} \textrm{ .}
\end{equation}
$\delta f$ PIC formulation:
\begin{eqnarray}
	\frac{\md x_{j_\alpha}}{\md t} & = & v_{j_\alpha} \textrm{ ,} \label{eq:particle_evolution_begin} \\
	\frac{\md v_{j_\alpha}}{\md t} & = & \frac{Z_\alpha}{m_\alpha} E(x_{j_\alpha}) \textrm{ ,} \\
	\frac{\md p_{j_\alpha}}{\md t} & = & 0 \textrm{ ,} \\
	\frac{\md w_{j_\alpha}}{\md t} & = & - (p_{j_\alpha} - w_{j_\alpha}) \frac{Z_\alpha}{m_\alpha} E(x_{j_\alpha}) \frac{\partial_v f_{0 \alpha}}{f_{0 \alpha}} \nonumber \\
	& = & (p_{j_\alpha} - w_{j_\alpha}) \frac{Z_\alpha}{T_\alpha} E(x_{j_\alpha}) v_{j_\alpha} \textrm{ ,} \label{eq:particle_evolution_end} \\
	\partial_x E(x_{j_x}) & = & 4 \pi \sum_\alpha \sum_{j_\alpha} w_{j_\alpha} S_x(x_{j_x} - x_{j_\alpha}) \textrm{ ,} \label{eq:poisson_df_pic} \\
	E(x_{j_\alpha}) & = & \sum_{j_x = 0}^{N_x - 1} E(x_{j_x}) S_x(x_{j_x} - x_{j_\alpha}) h_x \textrm{ .} \label{eq:field_scatter}
\end{eqnarray}
$j_x$ is index for grids in $x$.
Note that $j_x$ starts from $0$ for convenience to write the discrete Fourier transform (DFT) equations in the next paragraph.

In this model, $x$ is periodic and the box size is $l$.
\autoref{eq:poisson_df_pic} is solved by spectral method.
The charge density is given by:
\begin{equation}
	\rho(x_{j_x}) = \sum_\alpha \sum_{j_\alpha} w_{j_\alpha} S_x(x_{j_x} - x_{j_\alpha}) \textrm{ .} \label{eq:charge_density}
\end{equation}
Fourier expand charge density and electrostatic field:
\begin{equation}
	\left\{ \begin{array}{c}
		\rho(x_{j_x}) \\
		E(x_{j_x})
	\end{array} \right\} = \sum_{j_k = 0}^{N_x - 1} \left\{ \begin{array}{c}
		\rho_{j_k} \\
		E_{j_k}
	\end{array} \right\} \exp \left( i \frac{2 \pi}{N_x} j_k j_x \right) \textrm{ .} \label{eq:fourier_expansion}
\end{equation}
%Superscript $^*$ indicates complex conjugate.
The Fourier component coefficients are given by (DFT):
\begin{equation}
	\left\{ \begin{array}{c}
		\rho_{j_k} \\
		E_{j_k}
	\end{array} \right\} = \frac{1}{N_x} \sum_{j_k = 0}^{N_x - 1} \left\{ \begin{array}{c}
		\rho(x_{j_x}) \\
		E(x_{j_x})
	\end{array} \right\} \exp \left( -i \frac{2 \pi}{N_x} j_k j_x \right) \textrm{ .} \label{eq:dft}
\end{equation}
Plug Eqs.~\ref{eq:charge_density} and \ref{eq:fourier_expansion} into \autoref{eq:poisson_df_pic} and consider the orthogonality of Fourier harmonics to get:
\begin{eqnarray}
	i \frac{2 \pi}{l} j_k E_{j_k} & = & 4 \pi \rho_{j_k} \textrm{ ,} \label{eq:poisson_df_pic_fourier1} \\
	E_{j_k} & = & -i \frac{l}{2 \pi j_k} 4 \pi \rho_{j_k} \textrm{ .} \label{eq:poisson_df_pic_fourier2}
\end{eqnarray}


%%%%%%%%%%%%%%%%%%%%%%%%%%%%%%%%%%%%%%%%
\section{Normalization}

In \texttt{PIC1D-PETSc}, charge is normalized by proton charge $e$.
Mass is normalized by electron mass $m_{\mathrm{e}}$.
Temperature is normalized by electron temperature $T_{\mathrm{e}}$.
Time is normalized by the reciprocal of electron oscillation frequency $1/\omega_{\mathrm{pe}}$.
Length ($l$ or $x$) is normalized by electron Debye length $\lambda_{\mathrm{De}}$.
As a result, velocity is normalized by electron thermal velocity $v_{\mathrm{th},\mathrm{e}}$.
Physical particle distribution function ($f$ or $\delta f$) is normalized by electron density over thermal velocity $n_{0 \mathrm{e}} / v_{\mathrm{th},\mathrm{e}}$.
Physical particle density ($\delta n$) is normalized by electron density $n_{0 \mathrm{e}}$.
Electrostatic field is normalized by $T_{\mathrm{e}} / (e \lambda_{\mathrm{De}})$.

The normalized equations have the same form as described in \autoref{sec:df_pic}, except that the Poisson equation Eqs. \ref{eq:poisson_df_pic}, \ref{eq:poisson_df_pic_fourier1} and \ref{eq:poisson_df_pic_fourier2} no longer have the $4 \pi$ coefficient.


%%%%%%%%%%%%%%%%%%%%%%%%%%%%%%%%%%%%%%%%
\section{Vector-matrix form of $\delta f$ PIC formulation}

As this is a 1D model, all physical quantities are scalars.
All vectors mentioned in this section are mathematical vectors.
For each species, all particles' $x$ coordinates form a vector of size $N_\alpha$:
\begin{equation}
	\boldsymbol{x}_\alpha \equiv \left[ \begin{array}{c}
		x_{1,\alpha} \\
		x_{2,\alpha} \\
		\vdots \\
		x_{j_\alpha} \\
		\vdots \\
		x_{N_\alpha}
	\end{array} \right] \textrm{ .}
\end{equation}
Similarly, velocities and weights form vectors of $\boldsymbol{v}_\alpha$, $\boldsymbol{p}_\alpha$ and $\boldsymbol{w}_\alpha$, all of which are of size $N_\alpha$.

Field quantities $E$ and $\rho$ form vectors $\boldsymbol{E}$ and $\boldsymbol{\rho}$ of size $N_x$, respectively.
For example, $\boldsymbol{E}$ is given by:
\begin{equation}
	\boldsymbol{E} \equiv \left[ \begin{array}{c}
		E(x_{1}) \\
		E(x_{2}) \\
		\vdots \\
		E(x_{j_x}) \\
		\vdots \\
		E(x_{N_x})
	\end{array} \right] \textrm{ .}
\end{equation}

The particle evolution equations, i.e., Eqs.~\ref{eq:particle_evolution_begin}--\ref{eq:particle_evolution_end}, become:
\begin{eqnarray}
	\frac{\md \boldsymbol{x}_\alpha}{\md t} & = & \boldsymbol{v}_\alpha \textrm{ ,} \\
	\frac{\md \boldsymbol{v}_\alpha}{\md t} & = & \frac{Z_\alpha}{m_\alpha} \boldsymbol{E}_\alpha \textrm{ ,} \\
	\frac{\md \boldsymbol{p}_\alpha}{\md t} & = & 0 \textrm{ ,} \\
	\frac{\md \boldsymbol{w}_\alpha}{\md t} & = & -(\boldsymbol{p} - \boldsymbol{w}) \frac{Z_\alpha}{T_\alpha} \mathrm{diag} (\boldsymbol{E}_\alpha) \boldsymbol{v}_\alpha \textrm{ ,}
\end{eqnarray}
where $\mathrm{diag}()$ transforms a vector to a diagonal matrix:
\begin{equation}
	\mathrm{diag}(\boldsymbol{y}) = \mathrm{diag}( \left[ \begin{array}{c}
		y_1 \\
		y_2 \\
		\vdots \\
		y_N
	\end{array} \right] ) \equiv \left[ \begin{array}{cccc}
		y_1 & & & \\
		 & y_2 & & \\
		 & & \ddots & \\
		 & & & y_N
	\end{array} \right] \textrm{ ;}
\end{equation}
$\boldsymbol{E}_\alpha$ is a vector of size $N_\alpha$, and each element is the electrostatic field at the corresponding particle's position:
\begin{equation}
	\boldsymbol{E}_\alpha = \left[ \begin{array}{c}
		E(x_{1,\alpha}) \\
		E(x_{2,\alpha}) \\
		\vdots \\
		E(x_{j_\alpha}) \\
		\vdots \\
		E(x_{N_\alpha})
	\end{array} \right] \textrm{ .}
\end{equation}
The relation between $\boldsymbol{E}_\alpha$ and $\boldsymbol{E}$ can be found from \autoref{eq:field_scatter}, and is then given by:
\begin{equation}
	\boldsymbol{E}_\alpha = \mathbb{S}_\alpha \boldsymbol{E} \textrm{ ,}
\end{equation}
where $\mathbb{S}_\alpha$ is the shape matrix of size $N_\alpha \times N_x$ and its elements are given by:
\begin{equation}
	(\mathbb{S}_\alpha)_{j_\alpha, j_x} = S_x(x_{j_x} - x_{j_\alpha}) \cdot h_x = \left\{ \begin{array}{ll}
		1 - \frac{|x_{j_x} - x_{j_\alpha}|}{h_x} & |x_{j_x} - x_{j_\alpha}| < h_x \\
		0 & \textrm{otherwise}
	\end{array} \right. \textrm{ .}
\end{equation}
Note that $\mathbb{S}_\alpha$ is a sparse matrix.
For the hat shape function, each row of $\mathbb{S}_\alpha$ has two non-zero elements.
The charge density vector $\boldsymbol{\rho}$ of size $N_x$ can be found from \autoref{eq:charge_density}, and is then given by:
\begin{equation}
	\boldsymbol{\rho} = \sum_\alpha Z_\alpha \frac{\mathbb{S}_\alpha^{\mathrm{T}} \boldsymbol{w}_\alpha}{h_x} \textrm{ ,}
\end{equation}
where superscript $^\mathrm{T}$ indicates matrix transpose.

Because DFT is a linear transform, the DFT of charge density \autoref{eq:dft} can be written as:
\begin{equation}
	\boldsymbol{\rho}_k = \frac{1}{N_x} \mathbb{F} \boldsymbol{\rho} \textrm{ ,} \label{eq:dft_vm}
\end{equation}
where $\mathbb{F}$ is the DFT operator matrix of size $N_x \times N_x$, and its elements are given by:
\begin{equation}
	(\mathbb{F})_{j_k,j_x} = \exp \left( -i \frac{2 \pi}{N_x} j_k j_x \right) \textrm{ .}
\end{equation}
Note that literally performing the matrix multiplication in \autoref{eq:dft_vm} has a complexity $O(N_x^2)$.
When the whole spectrum is needed, fast Fourier transform (FFT) should be used instead, which reduces the complexity to $O(N_x \log_2 N_x)$ if $N_x$ is a power of $2$.
If only a few Fourier components are needed, then only the corresponding rows in $\mathbb{F}$ need to be kept, which is implemented in \texttt{PIC1D-PETSc}.
Then the electrostatic field in Fourier space can be found from \autoref{eq:poisson_df_pic_fourier2}, and the vector-matrix form is given by (after normalization, the factor $4 \pi$ does not show up here):
\begin{equation}
	\boldsymbol{E}_k = -i \cdot \mathrm{diag}(\boldsymbol{k})^{-1} \boldsymbol{\rho}_k \textrm{ ,} \label{eq:poisson_df_pic_fourier2_vm}
\end{equation}
where vector $k$ in full-spectrum case is given by:
\begin{equation}
	\boldsymbol{k} = \left[ \begin{array}{c}
		0 \\
		\frac{2 \pi}{l} \\
		\frac{2 \pi}{l} \cdot 2 \\
		\vdots \\
		\frac{2 \pi}{l} \cdot (N_x - 1)
	\end{array} \right] \textrm{ .}
\end{equation}
If only some Fourier components are kept, then the size of $\boldsymbol{k}$ becomes the number of the kept components, and only the corresponding elements show up in $\boldsymbol{k}$.
Note that the first element of $\boldsymbol{k}$ is 0, which corresponds to the $k = 0$ zonal component.
This would not be a problem when zonal component is excluded from the system.
If there is a case when zonal component needs to be kept, it needs special care and cannot be put in $\boldsymbol{E}_k$ and $\boldsymbol{k}$ directly as the matrix inversion $\mathrm{diag}(\boldsymbol{k})^{-1}$ would fail. Finally, the electrostatic field on the real space grid is given by the inverse DFT:
\begin{equation}
	\boldsymbol{E} = \mathbb{F}^\dagger \boldsymbol{E}_k \textrm{ ,}
\end{equation}
where $^\dagger$ indicates conjugate transpose.

\end{document}

